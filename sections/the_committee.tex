\section{The Committee}
\subsection{Powers of the committee}
\begin{enumerate}
  \item The committee is to be called the committee of management of the Association and, subject to the Act, the Regulation and these rules and to any resolution passed by the Association in General Meeting:
    \begin{enumerate}
      \item is to control and manage the affairs of the Association, and
      \item may exercise all such functions as may be exercised by the Association, other than those functions that are required by these rules to be exercised by a General Meeting of members of the Association, and
      \item has power to perform all such acts and do all such things as appear to the committee to be necessary or desirable for the proper management of the affairs of the Association.
    \end{enumerate}
  \item The committee shall not delegate all its powers to one role, person, or sub-committee, for example Artistic Director or Production Manager.
  \item The committee will appoint a Director, Musical Director, Choreographer \& Production Manager for each production. The Director, Musical Director and Production Manager will have their appointment formalised by written contract specifying their roles and responsibilities.
  \item A Production Team will be formed at the discretion of the Director and Production Manager, with the committee being informed of all appointments to production team roles.
  \item
    \begin{enumerate}
      \item Once the Production Team has been formed, the committee shall not interfere with the direction of the production unless, as passed by a unanimous vote at a meeting of the committee reaching quorum, the committee believes the integrity of the production and/or the reputation of the Association is being jeopardised, and
      \item if the committee reaches this decision, it shall also have the power to remove, replace and add production team members as it sees fit.
    \end{enumerate}
\end{enumerate}
\subsection{Constitution and membership}
\begin{enumerate}
  \item Subject in the case of the first members of the committee to section 21 of the Act, the committee is to consist of a minimum of:
    \begin{enumerate}
      \item the office-bearers of the Association
        \begin{enumerate}
          \item President
          \item Vice-president
          \item Reasurer
          \item Secretary
        \end{enumerate}
      \item and 3 ordinary members, each of whom is to be elected at the annual General Meeting of the Association under rule 15.
    \end{enumerate}
  \item Additional committee members may be elected at the Annual General Meeting to fulfil specific functions that may be required.
  \item Each member of the committee is, subject to these rules, to hold office until the conclusion of the Annual General Meeting following the date of the member’s election, but is eligible for re-election.
  \item In the event of a casual vacancy occurring in the membership of the committee, the committee may appoint a member of the Association to fill the vacancy and the member so appointed is to hold office, subject to these rules, until the conclusion of the Annual General Meeting next following the date of the appointment.
\end{enumerate}
\subsection{Election of members}
\begin{enumerate}
  \item Nominations of candidates for election as office-bearers of the Association or as ordinary members of the committee:
    \begin{enumerate}
      \item may be made in writing, signed by 2 members of the Association and accompanied by the written consent of the candidate (which may be endorsed on the form of the nomination);  such nomination must be delivered to the secretary of the Association at least 7 days before the date fixed for the holding of the Annual General Meeting at which the election is to take place, or
      \item may be made by two members of the Association at the Annual General Meeting with the verbal consent of the candidate.
    \end{enumerate}
  \item If insufficient nominations are received to fill all vacancies on the committee, the candidates nominated are taken to be elected and further nominations are to be received at the Annual General Meeting.
  \item If insufficient further nominations are received, any vacant positions remaining on the committee are taken to be casual vacancies.
  \item If the number of nominations received is equal to the number of vacancies to be filled, the persons nominated are taken to be elected.
  \item If the number of nominations received exceeds the number of vacancies to be filled, a ballot is to be held.
  \item The ballot for the election of office-bearers and ordinary members of the committee is to be conducted at the Annual General Meeting in such usual and proper manner as the committee may direct.
\end{enumerate}
\subsection{Office Bearers}
\begin{enumerate}
  \item President
    The president of the Association:
    \begin{enumerate}
      \item Presides over all committee meetings ensuring that the proceedings are in accordance with the Rules of the Association.
      \item Presides over all General Meetings of the Association.
      \item Is familiar with the functions of all office bearers to ensure that the functions are carried out effectively.
      \item Prepares and delivers the annual President’s Report to the Annual General Meeting.
    \end{enumerate}
  \item Vice-President
    The Vice-President of the Association:
    \begin{enumerate}
      \item Assists the President in carrying out his/her functions.
      \item Is familiar with all ongoing activities so as to be able to deputise for the President.
    \end{enumerate}
  \item Secretary
    The secretary of the Association must as soon as practicable after being appointed as
    Secretary lodge notice with the Association of his or her address.
    \begin{enumerate}
      \item It is the duty of the secretary to keep minutes and records of:
        \begin{enumerate}
          \item all appointments of office-bearers and members of the committee,
          \item the names of members of the committee present at a committee meeting or a general meeting
          \item all proceedings at committee meetings and General Meetings,
          \item all correspondence.
        \end{enumerate}
      \item Minutes of proceedings at a meeting must be signed by the chairperson of the meeting or by the chairperson of the next succeeding meeting.
    \end{enumerate}
  \item Treasurer
    \begin{enumerate}
      \item It is the duty of the treasurer of the Association to ensure:
        \begin{enumerate}
          \item that all money due to the Association is collected and received and that all payments authorised by the Association are made, and
          \item that correct books and accounts are kept showing the financial affairs of the Association, including full details of all receipts and expenditure connected with the activities of the Association, and
        \end{enumerate}
      \item At the Annual General Meeting the Treasurer must present a financial report showing the profit/loss account for each production and a profit/loss account for the Association’s full financial year.
      \item Within six weeks of the conclusion of a production the Treasurer must present a detailed financial report to the committee, particularly as compared with budget.
    \end{enumerate}
\end{enumerate}
\subsection{Cheque Signatories}
Signatories to the Association’s cheques may be any two of the following:
\begin{itemize}
  \item President
  \item Vice-President
  \item Secretary
  \item Treasurer
\end{itemize}
\subsection{Casual vacancies}
For the purposes of these rules, a casual vacancy in the office of a member of the committee occurs if the member:
\begin{enumerate}
  \item dies, or
  \item ceases to be a member of the Association, or
  \item becomes an insolvent under administration within the meaning of the Corporations Act 2001 of the Commonwealth, or
  \item resigns office by notice in writing given to the secretary, or
  \item is removed from office under rule 19, or
  \item becomes a mentally incapacitated person, or
  \item is absent without the consent of the committee from three consecutive meetings.
\end{enumerate}
\subsection{Removal of member}
\begin{enumerate}
  \item The Association in General Meeting may by resolution remove any member of the committee from the office of member before the expiration of the member’s term of office and may by resolution appoint another person to hold office until the expiration of the term of office of the member so removed.
  \item If a member of the committee to whom a proposed resolution referred to in clause (1) relates makes representations in writing to the secretary or president (not exceeding a reasonable length) and requests that the representations be notified to the members of the Association, the secretary or the president may send a copy of the representations to each member of the Association or, if the representations are not so sent, the member is entitled to require that the representations be read out at the meeting at which the resolution is considered.
\end{enumerate}
\subsection{Meetings and quorum}
\label{meetings-and-quorum}
\begin{enumerate}
  \item \begin{enumerate}
    \item The committee must meet at least once a month at such place and time as the committee may determine, taking into account the rehearsals of the Association, thus ensuring that committee members are able to participate in productions (see Rule 13 (5)).
    \item Committee members not performing in the current production must make themselves available for committee duties at rehearsals, as required.
  \end{enumerate}
\item Additional meetings of the committee may be convened by the president or by any member of the committee.
\item Oral or written notice of a meeting of the committee must be given by the secretary to each member of the committee at least 48 hours (or such other period as may be unanimously agreed on by the members of the committee) before the time appointed for the holding of the meeting.
\item Notice of a meeting given under clause (3) must specify the general nature of the business to be transacted at the meeting and no business other than that business is to be transacted at the meeting, except business which the committee members present at the meeting unanimously agree to treat as urgent business.
\item \label{five-members}Any five members of the committee constitute a quorum for the transaction of the business of a meeting of the committee.
\item No business is to be transacted by the committee unless a quorum is present and if, within half an hour of the time appointed for the meeting, a quorum is not present, the meeting is to stand adjourned to the same place and at the same hour of the same day in the following week.
\item If at the adjourned meeting a quorum is not present within half an hour of the time appointed for the meeting, the meeting is to be dissolved.
\item At a meeting of the committee:
  \begin{enumerate}
    \item the president or, in the president’s absence, the vice-president is to preside, or
    \item if the president and the vice-president are absent or unwilling to act, such one of the remaining members of the committee as may be chosen by the members present at the meeting is to preside.
  \end{enumerate}
\item Unless otherwise determined by the committee, the following will be the order of business at each meeting of the committee:
  \begin{enumerate}
    \item Apologies
    \item Minutes of the last meeting and business arising from those minutes
    \item Correspondence and business arising thereupon
    \item Treasurer’s report and accounts for payment
    \item New business
  \end{enumerate}
\end{enumerate}
\subsection{Delegation by committee to sub-committee}
\begin{enumerate}
  \item The committee may, by instrument in writing, delegate to one or more sub-committees (consisting of such member or members of the Association as the committee thinks fit) the exercise of such of the functions of the committee as are specified in the instrument, other than:
    \begin{enumerate}
      \item this power of delegation, and
      \item a function which is a duty imposed on the committee by the Act or by any other law.
    \end{enumerate}
  \item A function the exercise of which has been delegated to a sub-committee under this rule may, while the delegation remains unrevoked, be exercised from time to time by the sub-committee in accordance with the terms of the delegation.
  \item A delegation under this section may be made subject to such conditions or limitations as to the exercise of any function, or as to time or circumstances, as may be specified in the instrument of delegation.
  \item Despite any delegation under this rule, the committee may continue to exercise any function delegated.
  \item Any act or thing done or suffered by a sub-committee acting in the exercise of a delegation under this rule has the same force and effect as it would have if it had been done or suffered by the committee.
  \item The committee may, by instrument in writing, revoke wholly or in part any delegation under this rule.
  \item A sub-committee may meet and adjourn as it thinks proper.
  \item At least one member of the committee must be a member of any sub-committee.
\end{enumerate}
\subsection{Voting and decisions}
\begin{enumerate}
  \item Questions arising at a meeting of the committee or of any sub-committee appointed by the committee are to be determined by a majority of the votes of members of the committee or sub-committee present at the meeting.
  \item Each member present at a meeting of the committee or of any sub-committee appointed by the committee (including the person presiding at the meeting) is entitled to one vote but, in the event of an equality of votes on any question, the person presiding may exercise a second or casting vote.
  \item Subject to rule \ref{meetings-and-quorum} (\ref{five-members}), the committee may act despite any vacancy on the committee.
  \item Any act or thing done or suffered, or purporting to have been done or suffered, by the committee or by a sub-committee appointed by the committee, is valid and effectual despite any defect that may afterwards be discovered in the appointment or qualification of any member of the committee or sub-committee.
  \item Should there be a conflict of interest between a subject under discussion and a member/members of the committee (for example the appointment of a committee member or a relative of a committee member/members to the Production Team), that member/those members must leave the meeting until the discussion is concluded.
\end{enumerate}
