\documentclass{article}
\usepackage[utf8]{inputenc}
\usepackage[a4paper, margin=1in]{geometry}
\usepackage{fancyhdr}
\usepackage{xcolor}
\usepackage{hyperref}
\usepackage{lastpage}

\title{Manly Musical Society Constitution}
\author{Linus Karsai (Secretary)}
\date{\today}

\pagestyle{fancy}
\fancyhf{}
\fancyhead[RE,LO]{Manly Musical Society Constitution}
\fancyhead[LE,RO]{\leftmark}
\fancyfoot[R]{\thepage \ of \pageref{LastPage}}

\begin{document}

\maketitle

\tableofcontents
\clearpage
\section{Preliminary}

\subsection{Name of Association}
The Association shall be called Manly Musical Society Inc (hereinafter referred to as the Association), also referred to as MMS.
\subsection{Object of the Association}
The object of Manly Musical Society Inc is the presentation of musical theatre.  The Association will achieve this by the cultivation of vocal music by practice and subsequent public performance.  The Association is not carried on for the personal gain of any member or members.  Any profits derived from performances of the Association will be paid into the Association’s funds for furtherance of the aims and object of the Association or for donation to any charity.
\subsection{Registered Address}
The registered address of the Association is the residential address of the Public Officer of the Association.
\subsection{Definitions}
In these rules, the following roles are defined as:
\begin{enumerate}
  \item \textit{Director-General} means the Director-General of the Department of Fair Trading.
  \item \textit{Ordinary member} member means a member of the committee who is not an office-bearer of the Association, as referred to in rule 14 (2).
  \item \textit{Secretary} means:
    \begin{enumerate}
      \item the person holding office under these rules as secretary of the Association, or
      \item if no such person holds that office—the public officer of the Association.
    \end{enumerate}
  \item \textit{Public officer} means:
    \begin{enumerate}
      \item the person appointed by the Association under the rules of the Association.
    \end{enumerate}
\end{enumerate}
In these rules, the following terms are defined as:
\begin{enumerate}
  \item \textit{The Act} means the \underline{\href{https://legislation.nsw.gov.au/\#/view/act/1984/143/full}{Associations Incorporation Act 1984}}.
  \item \textit{The Regulation} means the \underline{\href{https://www.legislation.nsw.gov.au/\#/view/subordleg/1999/484}{Associations Incorporation Regulation 1999}}.
\end{enumerate}
In these rules:
\begin{enumerate}
  \item A reference to a function includes a reference to a power, authority and duty.
  \item A reference to the exercise of a function includes, if the function is a duty, a reference to the performance of the duty.
  \item The provisions of the Interpretation Act 1987 apply to and in respect of these rules in the same manner as those provisions would so apply if these rules were an instrument made under the Act.
\end{enumerate}

\section{Membership}
\subsection{Membership qualifications}
A person is qualified to be a member of the Association if, but only if:
\begin{enumerate}
  %TODO: Fix references
  \item the person is a person referred to in section 15 (1) (a), (b) or (c) of the Act and has not ceased to be a member of the Association at any time after incorporation of the Association under the Act, or
  \item the person is a natural person who:
    \begin{enumerate}
      \item has completed and signed an Application for Membership form agreeing to the rules
        and requirements of the Association, and
      \item has been approved for membership of the Association by the committee of the Association, and
      \item has paid the membership joining fee and annual membership fee within the period of two months after receipt of notification of acceptance;
    \end{enumerate}
\end{enumerate}
\subsection{Application of membership}
\begin{enumerate}
  \item \begin{enumerate}
    \item Application of a person for membership of the Association must be made in writing in the form set out in Appendix 1 to these rules, and
    \item must be lodged with the secretary of the Association.
  \end{enumerate}
\item As soon as practicable after receiving an application for membership, the secretary must refer the application to the committee which is to determine whether to approve or to reject the application.
\item As soon as practicable after the committee makes that determination, the secretary must:
  \begin{enumerate}
    \item notify the applicant that the committee approved or rejected the application (whichever is applicable), and
    \item \label{application-payment} if the committee approved the application, request the applicant to pay (within the period of two months after receipt by the applicant of the notification) the sum payable under these rules by a member as joining fee and annual subscription.
  \end{enumerate}
\item The secretary must, on payment by the applicant of the amounts referred to in clause \ref{application-payment} within the period referred to in that provision, enter the applicant’s name in the register of members and, on the name being so entered, the applicant becomes a member of the Association.
\item \begin{enumerate}
  \item A member may submit the name of any member who has rendered exceptional service to the Association for election at an Annual General Meeting as a Life Member of the association.
  \item A Life Member of the committee shall not have to pay any membership or annual fee, and may have any extra provision allotted to them by the committee as it sees fit.
\end{enumerate}
\end{enumerate}
\subsection{Cessation of membership}
A person ceases to be a member of the Association if the person:
\begin{enumerate}
  \item dies, or
  \item resigns membership, or
  \item is expelled from the Association, or
  \item fails to pay the annual membership fee by the due date.
\end{enumerate}
\subsection{Membership entitlements not transferable}
A right, privilege or obligation which a person has by reason of being a member of the Association:
\begin{enumerate}
  \item is not capable of being transferred or transmitted to another person, and
  \item terminates on cessation of the person’s membership.
\end{enumerate}

\subsection{Resignation of membership}
\begin{enumerate}
  \item A member of the Association is not entitled to resign that membership except in accordance with this rule.
  \item A member of the Association who has paid all amounts payable by the member to the Association in respect of the member’s membership may resign from membership of the Association by first giving to the secretary written notice of at least one month (or such other period as the committee may determine) of the member’s intention to resign and, on the expiration of the period of notice, the member ceases to be a member.
  \item If a member of the Association ceases to be a member under clause 2, and in every other case where a member ceases to hold membership, the secretary must make an appropriate entry in the register of members recording the date on which the member ceased to be a member.
\end{enumerate}
\subsection{Register of members}
\begin{enumerate}
  \item The public officer of the Association must establish and maintain a register of members of the Association specifying the name and address of each person who is a member of the Association together with the date on which the person became a member.
  \item The register of members must be kept at the principal place of administration of the Association and must be open for inspection, free of charge, by any member of the Association at any reasonable hour, subject to the provisions of the Privacy Act.
  \item A member of the Association may obtain a copy of any part of the register on payment of a fee of \$1 for each page copied or, if some other amount is determined by the committee, that other amount, subject to the provisions of the Privacy Act.
\end{enumerate}
\subsection{Fees and subscriptions}
\begin{enumerate}
  \item A member of the Association must, on admission to membership, pay to the Association a joining fee of \$5 which will be deemed to be part of the first year’s membership fee.
  \item A member must pay to the Association an annual membership fee of \$25 or, if some other amount is determined by the committee, that other amount within two months of joining the Association:
  \item The Association’s full financial year is 1 January to 31 December.
  \item Performing members must pay, in addition to amounts payable in clauses (1) and (2), a fee for each show in which they participate, as determined by the committee.  This must be paid no later than one month after the first rehearsal.
  \item At the discretion of the committee, the fees of an individual may be waived or reduced. A person whose fees have been waived or reduced by the committee has no voting rights.
\end{enumerate}
\subsection{Members' liabilities}
The liability of a member of the Association to contribute towards the payment of the debts and liabilities of the Association or the costs, charges and expenses of the winding up of the Association is limited to the amount, if any, unpaid by the member in respect of membership of the Association as required by rule 8.
\subsection{Resolution of internal disputes}
\begin{enumerate}
  \item Disputes between members (in their capacity as members) of the Association, and disputes between members and the Association, are to be referred to a community justice centre for mediation in accordance with the \underline{\href{https://legislation.nsw.gov.au/\#/view/act/1983/127}{Community Justice Centres Act 1983}}.
  \item At least 7 days before a mediation session is to commence, the parties are to exchange statements of the issues that are in dispute between them and supply copies to the mediator.
\end{enumerate}
\subsection{Disciplining of members}
\begin{enumerate}
  \item A complaint may be made to the committee by any person that is a member of the Association:
    \begin{enumerate}
      \item has persistently refused or neglected to comply with a provision or provisions of these rules, or
      \item has persistently and wilfully acted in a manner prejudicial to the interests of the Association.
    \end{enumerate}
  \item On receiving such a complaint, the committee:
    \begin{enumerate}
      \item must cause notice of the complaint to be served on the member concerned, and
      \item must give the member at least 14 days from the time the notice is served within which to make submissions to the committee in connection with the complaint, and
      \item must take into consideration any submissions made by the member in connection with the complaint.
    \end{enumerate}
  \item The committee may, by resolution, expel the member from the Association or suspend the member from membership of the Association if, after considering the complaint and any submissions made in connection with the complaint, it is satisfied that the facts alleged in the complaint have been proved.
  \item The committee may, by resolution, exclude a performing member from participating in a production if that member fails to attend three consecutive rehearsals without the approval of the production team.
  \item If the committee expels or suspends a member, the secretary must, within 7 days after the action is taken, cause written notice to be given to the member of the action taken, of the reasons given by the committee for having taken that action and of the member’s right of appeal under rule 12.
  \item The expulsion or suspension does not take effect:
    \begin{enumerate}
      \item until the expiration of the period within which the member is entitled to appeal against the resolution concerned, or
      \item if within that period the member exercises the right of appeal, unless and until the Association confirms the resolution under rule 12 (5), whichever is the later.
    \end{enumerate}
\end{enumerate}
\subsection{Right of appeal of disciplined member}
\begin{enumerate}
  \item A member may appeal to the Association in General Meeting against a resolution of the committee under rule 11, within 7 days after notice of the resolution is served on the member, by lodging with the secretary a notice to that effect.
  \item The notice may, but need not, be accompanied by a statement of the grounds on which the member intends to rely for the purposes of the appeal.
  \item On receipt of a notice from a member under clause (1), the secretary must notify the committee which is to convene a General Meeting of the Association to be held within 28 days after the date on which the secretary received the notice.
  \item At a General Meeting of the Association convened under clause (3):
    \begin{enumerate}
      \item no business other than the question of the appeal is to be transacted, and
      \item the committee and the member must be given the opportunity to state their respective cases orally or in writing, or both, and
      \item the members present are to vote by secret ballot on the question of whether the resolution should be confirmed or revoked.
    \end{enumerate}
  \item If at the General Meeting the Association passes a special resolution in favour of the confirmation of the resolution, the resolution is confirmed.
\end{enumerate}
\section{The Committee}
\subsection{Powers of the committee}
\begin{enumerate}
  \item The committee is to be called the committee of management of the Association and, subject to the Act, the Regulation and these rules and to any resolution passed by the Association in General Meeting:
    \begin{enumerate}
      \item is to control and manage the affairs of the Association, and
      \item may exercise all such functions as may be exercised by the Association, other than those functions that are required by these rules to be exercised by a General Meeting of members of the Association, and
      \item has power to perform all such acts and do all such things as appear to the committee to be necessary or desirable for the proper management of the affairs of the Association.
    \end{enumerate}
  \item The committee shall not delegate all its powers to one role, person, or sub-committee, for example Artistic Director or Production Manager.
  \item The committee will appoint a Director, Musical Director, Choreographer \& Production Manager for each production. The Director, Musical Director and Production Manager will have their appointment formalised by written contract specifying their roles and responsibilities.
  \item A Production Team will be formed at the discretion of the Director and Production Manager, with the committee being informed of all appointments to production team roles.
  \item
    \begin{enumerate}
      \item Once the Production Team has been formed, the committee shall not interfere with the direction of the production unless, as passed by a unanimous vote at a meeting of the committee reaching quorum, the committee believes the integrity of the production and/or the reputation of the Association is being jeopardised, and
      \item if the committee reaches this decision, it shall also have the power to remove, replace and add production team members as it sees fit.
    \end{enumerate}
\end{enumerate}
\subsection{Constitution and membership}
\begin{enumerate}
  \item Subject in the case of the first members of the committee to section 21 of the Act, the committee is to consist of a minimum of:
    \begin{enumerate}
      \item the office-bearers of the Association
        \begin{enumerate}
          \item President
          \item Vice-president
          \item Reasurer
          \item Secretary
        \end{enumerate}
      \item and 3 ordinary members, each of whom is to be elected at the annual General Meeting of the Association under rule 15.
    \end{enumerate}
  \item Additional committee members may be elected at the Annual General Meeting to fulfil specific functions that may be required.
  \item Each member of the committee is, subject to these rules, to hold office until the conclusion of the Annual General Meeting following the date of the member’s election, but is eligible for re-election.
  \item In the event of a casual vacancy occurring in the membership of the committee, the committee may appoint a member of the Association to fill the vacancy and the member so appointed is to hold office, subject to these rules, until the conclusion of the Annual General Meeting next following the date of the appointment.
\end{enumerate}
\subsection{Election of members}
\begin{enumerate}
  \item Nominations of candidates for election as office-bearers of the Association or as ordinary members of the committee:
    \begin{enumerate}
      \item may be made in writing, signed by 2 members of the Association and accompanied by the written consent of the candidate (which may be endorsed on the form of the nomination);  such nomination must be delivered to the secretary of the Association at least 7 days before the date fixed for the holding of the Annual General Meeting at which the election is to take place, or
      \item may be made by two members of the Association at the Annual General Meeting with the verbal consent of the candidate.
    \end{enumerate}
  \item If insufficient nominations are received to fill all vacancies on the committee, the candidates nominated are taken to be elected and further nominations are to be received at the Annual General Meeting.
  \item If insufficient further nominations are received, any vacant positions remaining on the committee are taken to be casual vacancies.
  \item If the number of nominations received is equal to the number of vacancies to be filled, the persons nominated are taken to be elected.
  \item If the number of nominations received exceeds the number of vacancies to be filled, a ballot is to be held.
  \item The ballot for the election of office-bearers and ordinary members of the committee is to be conducted at the Annual General Meeting in such usual and proper manner as the committee may direct.
\end{enumerate}
\subsection{Office Bearers}
\begin{enumerate}
  \item President
    The president of the Association:
    \begin{enumerate}
      \item Presides over all committee meetings ensuring that the proceedings are in accordance with the Rules of the Association.
      \item Presides over all General Meetings of the Association.
      \item Is familiar with the functions of all office bearers to ensure that the functions are carried out effectively.
      \item Prepares and delivers the annual President’s Report to the Annual General Meeting.
    \end{enumerate}
  \item Vice-President
    The Vice-President of the Association:
    \begin{enumerate}
      \item Assists the President in carrying out his/her functions.
      \item Is familiar with all ongoing activities so as to be able to deputise for the President.
    \end{enumerate}
  \item Secretary
    The secretary of the Association must as soon as practicable after being appointed as
    Secretary lodge notice with the Association of his or her address.
    \begin{enumerate}
      \item It is the duty of the secretary to keep minutes and records of:
        \begin{enumerate}
          \item all appointments of office-bearers and members of the committee,
          \item the names of members of the committee present at a committee meeting or a general meeting
          \item all proceedings at committee meetings and General Meetings,
          \item all correspondence.
        \end{enumerate}
      \item Minutes of proceedings at a meeting must be signed by the chairperson of the meeting or by the chairperson of the next succeeding meeting.
    \end{enumerate}
  \item Treasurer
    \begin{enumerate}
      \item It is the duty of the treasurer of the Association to ensure:
        \begin{enumerate}
          \item that all money due to the Association is collected and received and that all payments authorised by the Association are made, and
          \item that correct books and accounts are kept showing the financial affairs of the Association, including full details of all receipts and expenditure connected with the activities of the Association, and
        \end{enumerate}
      \item At the Annual General Meeting the Treasurer must present a financial report showing the profit/loss account for each production and a profit/loss account for the Association’s full financial year.
      \item Within six weeks of the conclusion of a production the Treasurer must present a detailed financial report to the committee, particularly as compared with budget.
    \end{enumerate}
\end{enumerate}
\subsection{Cheque Signatories}
Signatories to the Association’s cheques may be any two of the following:
\begin{itemize}
  \item President
  \item Vice-President
  \item Secretary
  \item Treasurer
\end{itemize}
\subsection{Casual vacancies}
For the purposes of these rules, a casual vacancy in the office of a member of the committee occurs if the member:
\begin{enumerate}
  \item dies, or
  \item ceases to be a member of the Association, or
  \item becomes an insolvent under administration within the meaning of the Corporations Act 2001 of the Commonwealth, or
  \item resigns office by notice in writing given to the secretary, or
  \item is removed from office under rule 19, or
  \item becomes a mentally incapacitated person, or
  \item is absent without the consent of the committee from three consecutive meetings.
\end{enumerate}
\subsection{Removal of member}
\begin{enumerate}
  \item The Association in General Meeting may by resolution remove any member of the committee from the office of member before the expiration of the member’s term of office and may by resolution appoint another person to hold office until the expiration of the term of office of the member so removed.
  \item If a member of the committee to whom a proposed resolution referred to in clause (1) relates makes representations in writing to the secretary or president (not exceeding a reasonable length) and requests that the representations be notified to the members of the Association, the secretary or the president may send a copy of the representations to each member of the Association or, if the representations are not so sent, the member is entitled to require that the representations be read out at the meeting at which the resolution is considered.
\end{enumerate}
\subsection{Meetings and quorum}
\label{meetings-and-quorum}
\begin{enumerate}
  \item \begin{enumerate}
    \item The committee must meet at least once a month at such place and time as the committee may determine, taking into account the rehearsals of the Association, thus ensuring that committee members are able to participate in productions (see Rule 13 (5)).
    \item Committee members not performing in the current production must make themselves available for committee duties at rehearsals, as required.
  \end{enumerate}
\item Additional meetings of the committee may be convened by the president or by any member of the committee.
\item Oral or written notice of a meeting of the committee must be given by the secretary to each member of the committee at least 48 hours (or such other period as may be unanimously agreed on by the members of the committee) before the time appointed for the holding of the meeting.
\item Notice of a meeting given under clause (3) must specify the general nature of the business to be transacted at the meeting and no business other than that business is to be transacted at the meeting, except business which the committee members present at the meeting unanimously agree to treat as urgent business.
\item \label{five-members}Any five members of the committee constitute a quorum for the transaction of the business of a meeting of the committee.
\item No business is to be transacted by the committee unless a quorum is present and if, within half an hour of the time appointed for the meeting, a quorum is not present, the meeting is to stand adjourned to the same place and at the same hour of the same day in the following week.
\item If at the adjourned meeting a quorum is not present within half an hour of the time appointed for the meeting, the meeting is to be dissolved.
\item At a meeting of the committee:
  \begin{enumerate}
    \item the president or, in the president’s absence, the vice-president is to preside, or
    \item if the president and the vice-president are absent or unwilling to act, such one of the remaining members of the committee as may be chosen by the members present at the meeting is to preside.
  \end{enumerate}
\item Unless otherwise determined by the committee, the following will be the order of business at each meeting of the committee:
  \begin{enumerate}
    \item Apologies
    \item Minutes of the last meeting and business arising from those minutes
    \item Correspondence and business arising thereupon
    \item Treasurer’s report and accounts for payment
    \item New business
  \end{enumerate}
\end{enumerate}
\subsection{Delegation by committee to sub-committee}
\begin{enumerate}
  \item The committee may, by instrument in writing, delegate to one or more sub-committees (consisting of such member or members of the Association as the committee thinks fit) the exercise of such of the functions of the committee as are specified in the instrument, other than:
    \begin{enumerate}
      \item this power of delegation, and
      \item a function which is a duty imposed on the committee by the Act or by any other law.
    \end{enumerate}
  \item A function the exercise of which has been delegated to a sub-committee under this rule may, while the delegation remains unrevoked, be exercised from time to time by the sub-committee in accordance with the terms of the delegation.
  \item A delegation under this section may be made subject to such conditions or limitations as to the exercise of any function, or as to time or circumstances, as may be specified in the instrument of delegation.
  \item Despite any delegation under this rule, the committee may continue to exercise any function delegated.
  \item Any act or thing done or suffered by a sub-committee acting in the exercise of a delegation under this rule has the same force and effect as it would have if it had been done or suffered by the committee.
  \item The committee may, by instrument in writing, revoke wholly or in part any delegation under this rule.
  \item A sub-committee may meet and adjourn as it thinks proper.
  \item At least one member of the committee must be a member of any sub-committee.
\end{enumerate}
\subsection{Voting and decisions}
\begin{enumerate}
  \item Questions arising at a meeting of the committee or of any sub-committee appointed by the committee are to be determined by a majority of the votes of members of the committee or sub-committee present at the meeting.
  \item Each member present at a meeting of the committee or of any sub-committee appointed by the committee (including the person presiding at the meeting) is entitled to one vote but, in the event of an equality of votes on any question, the person presiding may exercise a second or casting vote.
  \item Subject to rule \ref{meetings-and-quorum} (\ref{five-members}), the committee may act despite any vacancy on the committee.
  \item Any act or thing done or suffered, or purporting to have been done or suffered, by the committee or by a sub-committee appointed by the committee, is valid and effectual despite any defect that may afterwards be discovered in the appointment or qualification of any member of the committee or sub-committee.
  \item Should there be a conflict of interest between a subject under discussion and a member/members of the committee (for example the appointment of a committee member or a relative of a committee member/members to the Production Team), that member/those members must leave the meeting until the discussion is concluded.
\end{enumerate}
\section{General Meetings}
\subsection{Annual General Meetings—holding of}
\begin{enumerate}
  \item With the exception of the first Annual General Meeting of the Association, the Association must, at least once in each calendar year and within the period of 6 months after the expiration of each financial year of the Association, convene an Annual General Meeting of its members.
  \item The Association must hold its first Annual General Meeting:
    \begin{enumerate}
      \item within the period of 18 months after its incorporation under the Act, and
      \item within the period of 6 months after the expiration of the first financial year of the Association.
    \end{enumerate}
  \item Clauses (1) and (2) have effect subject to any extension or permission granted by the Director-General under section 26 (3) of the Act.
\end{enumerate}
\subsection{Annual General Meetings—calling of and business at}
\begin{enumerate}
  \item The Annual General Meeting of the Association is, subject to the Act and to rule 23, to be convened on such date and at such place and time as the committee thinks fit.
  \item In addition to any other business which may be transacted at an Annual General Meeting, the business of an Annual General Meeting is to include the following:
    \begin{enumerate}
      \item to confirm the minutes of the last preceding Annual General Meeting and of any Special General Meeting held since that meeting,
      \item to receive from the committee reports on the activities of the Association during the last preceding financial year,
      \item to elect office-bearers of the Association and ordinary members of the committee,
      \item to elect the Public Officer of the Association, and
      \item to receive and consider the statement which is required to be submitted to members under section 26 (6) of the Act.
    \end{enumerate}
  \item An Annual General Meeting must be specified as such in the notice convening it.
\end{enumerate}
\subsection{Special General Meetings—calling of}
\begin{enumerate}
  \item The committee may, whenever it thinks fit, convene a Special General Meeting of the Association.
  \item The committee must, on the requisition in writing of not less than 12 financial members, convene a Special General Meeting of the Association.
  \item A requisition of members for a Special General Meeting:
    \begin{enumerate}
      \item must state the purpose or purposes of the meeting, and
      \item must be signed by the members making the requisition, and
      \item must be lodged with the secretary, and
      \item may consist of several documents in a similar form, each signed by one or more of the members making the requisition.
    \end{enumerate}
  \item If the committee fails to convene a Special General Meeting to be held within 1 month after that date on which a requisition of members for the meeting is lodged with the secretary, any one or more of the members who made the requisition may convene a Special General Meeting to be held not later than 3 months after that date.
  \item A Special General Meeting convened by a member or members as referred to in clause (4) must be convened as nearly as is practicable in the same manner as General Meetings are convened by the committee and any member who consequently incurs expense is entitled to be reimbursed by the Association for any expense so incurred.
\end{enumerate}
\subsection{Notice}
\begin{enumerate}
  \item Except if the nature of the business proposed to be dealt with at a General Meeting requires a special resolution of the Association, the secretary must, at least 14 days before the date fixed for the holding of the General Meeting, give a notice to each member specifying the place, date and time of the meeting and the nature of the business proposed to be transacted at the meeting.
  \item If the nature of the business proposed to be dealt with at a General Meeting requires a special resolution of the Association, the secretary must, at least 21 days before the date fixed for the holding of the General Meeting, cause notice to be given to each member specifying, in addition to the matter required under clause (1), the intention to propose the resolution as a special resolution.
  \item No business other than that specified in the notice convening a General Meeting is to be transacted at the meeting except, in the case of an Annual General Meeting, business which may be transacted under rule 24 (2).
  \item A member desiring to bring any business before a General Meeting may give notice in writing of that business to the secretary who must include that business in the next notice calling a General Meeting given after receipt of the notice from the member.
\end{enumerate}
\subsection{Procedure}
\begin{enumerate}
  \item No item of business is to be transacted at a General Meeting unless a quorum of members entitled under these rules to vote is present during the time the meeting is considering that item.
  \item Eleven members present in person (being members entitled under these rules to vote at a General Meeting) constitute a quorum for the transaction of the business of a General Meeting.
  \item If within half an hour after the appointed time for the commencement of a General Meeting a quorum is not present, the meeting:
    \begin{enumerate}
      \item if convened on the requisition of members, is to be dissolved, and
      \item in any other case, is to stand adjourned to the same day in the following week at the same time and (unless another place is specified at the time of the adjournment by the person presiding at the meeting or communicated by written notice to members given before the day to which the meeting is adjourned) at the same place.
    \end{enumerate}
  \item If at the adjourned meeting a quorum is not present within half an hour after the time appointed for the commencement of the meeting, the members present (being not less than seven) is to constitute a quorum.
\end{enumerate}
\subsection{Presiding member}
\begin{enumerate}
  \item The president or, in the president`s absence, the vice-president, is to preside as chairperson at each General Meeting of the Association.
  \item If the president and the vice-president are absent or unwilling to act, the members present must elect one of their number to preside as chairperson at the meeting.
\end{enumerate}
\end{document}
