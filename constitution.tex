\documentclass{article}
\usepackage[utf8]{inputenc}
\usepackage[a4paper, margin=1in]{geometry}

\usepackage{xcolor}
\usepackage{hyperref}

\title{Manly Musical Society Constitution}
\author{Linus Karsai}
\date{\today}

\begin{document}

\maketitle

\tableofcontents
\clearpage
\section{Preliminary}

\subsection{Name of Association}
The Association shall be called Manly Musical Society Inc (hereinafter referred to as the Association), also referred to as MMS.
\subsection{Object of the Association}
The object of Manly Musical Society Inc is the presentation of musical theatre.  The Association will achieve this by the cultivation of vocal music by practice and subsequent public performance.  The Association is not carried on for the personal gain of any member or members.  Any profits derived from performances of the Association will be paid into the Association’s funds for furtherance of the aims and object of the Association or for donation to any charity.
\subsection{Registered Address}
The registered address of the Association is the residential address of the Public Officer of the Association. 
\subsection{Definitions}
In these rules, the following roles are defined as:
\begin{enumerate}
    \item \textit{Director-General} means the Director-General of the Department of Fair Trading.
    \item \textit{Ordinary member} member means a member of the committee who is not an office-bearer of the Association, as referred to in rule 14 (2).
    \item \textit{Secretary} means:
    \begin{enumerate}
        \item the person holding office under these rules as secretary of the Association, or
        \item if no such person holds that office—the public officer of the Association.
    \end{enumerate}
    \item \textit{Public officer} means:
        \begin{enumerate}
            \item the person appointed by the Association under the rules of the Association.
        \end{enumerate}
\end{enumerate}
In these rules, the following terms are defined as:
\begin{enumerate}
    \item \textit{The Act} means the \underline{\href{https://legislation.nsw.gov.au/\#/view/act/1984/143/full}{Associations Incorporation Act 1984}}.
    \item \textit{The Regulation} means the \underline{\href{https://www.legislation.nsw.gov.au/\#/view/subordleg/1999/484}{Associations Incorporation Regulation 1999}}.
\end{enumerate}
In these rules:
\begin{enumerate}
    \item A reference to a function includes a reference to a power, authority and duty.
    \item A reference to the exercise of a function includes, if the function is a duty, a reference to the performance of the duty.
    \item The provisions of the Interpretation Act 1987 apply to and in respect of these rules in the same manner as those provisions would so apply if these rules were an instrument made under the Act.
\end{enumerate}

\section{Membership}
\subsection{Membership qualifications}
A person is qualified to be a member of the Association if, but only if: 
\begin{enumerate}
    // TODO: Fix references
    \item the person is a person referred to in section 15 (1) (a), (b) or (c) of the Act and has not ceased to be a member of the Association at any time after incorporation of the Association under the Act, or
    \item the person is a natural person who: 
    \begin{enumerate}
        \item has completed and signed an Application for Membership form agreeing to the rules 
       and requirements of the Association, and
       \item has been approved for membership of the Association by the committee of the Association, and
       \item has paid the membership joining fee and annual membership fee within the period of two months after receipt of notification of acceptance;
    \end{enumerate}
\end{enumerate}
\subsection{Application of membership}
\begin{enumerate}
    \item \begin{enumerate}
        \item Application of a person for membership of the Association must be made in writing in the form set out in Appendix 1 to these rules, and 
        \item must be lodged with the secretary of the Association.
    \end{enumerate}
    \item As soon as practicable after receiving an application for membership, the secretary must refer the application to the committee which is to determine whether to approve or to reject the application.
    \item As soon as practicable after the committee makes that determination, the secretary must: 
    \begin{enumerate}
        \item notify the applicant that the committee approved or rejected the application (whichever is applicable), and
        \item \label{application-payment} if the committee approved the application, request the applicant to pay (within the period of two months after receipt by the applicant of the notification) the sum payable under these rules by a member as joining fee and annual subscription.
    \end{enumerate}
    \item The secretary must, on payment by the applicant of the amounts referred to in clause \ref{application-payment} within the period referred to in that provision, enter the applicant’s name in the register of members and, on the name being so entered, the applicant becomes a member of the Association.
    \item \begin{enumerate}
        \item A member may submit the name of any member who has rendered exceptional service to the Association for election at an Annual General Meeting as a Life Member of the association.
        \item A Life Member of the committee shall not have to pay any membership or annual fee, and may have any extra provision allotted to them by the committee as it sees fit.
    \end{enumerate}
\end{enumerate}
\subsection{Cessation of membership}
A person ceases to be a member of the Association if the person:
\begin{enumerate}
    \item dies, or
    \item resigns membership, or
    \item is expelled from the Association, or
    \item fails to pay the annual membership fee by the due date.
\end{enumerate}
\subsection{Membership entitlements not transferable}
A right, privilege or obligation which a person has by reason of being a member of the Association:
\begin{enumerate}
    \item is not capable of being transferred or transmitted to another person, and
    \item terminates on cessation of the person’s membership.
\end{enumerate}
\end{document}
