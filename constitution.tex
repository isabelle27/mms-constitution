\documentclass{article}
\usepackage[utf8]{inputenc}
\usepackage[a4paper, margin=1in]{geometry}
\usepackage{fancyhdr}
\usepackage{xcolor}
\usepackage{hyperref}
\usepackage{lastpage}

\title{Manly Musical Society Constitution}
\author{Linus Karsai}
\date{\today}

\pagestyle{fancy}
\fancyhf{}
\fancyhead[RE,LO]{Manly Musical Society Constitution}
\fancyhead[LE,RO]{\leftmark}
\fancyfoot[R]{\thepage \ of \pageref{LastPage}}

\begin{document}

\maketitle

\tableofcontents
\clearpage
\section{Preliminary}

\subsection{Name of Association}
The Association shall be called Manly Musical Society Inc (hereinafter referred to as the Association), also referred to as MMS.
\subsection{Object of the Association}
The object of Manly Musical Society Inc is the presentation of musical theatre.  The Association will achieve this by the cultivation of vocal music by practice and subsequent public performance.  The Association is not carried on for the personal gain of any member or members.  Any profits derived from performances of the Association will be paid into the Association’s funds for furtherance of the aims and object of the Association or for donation to any charity.
\subsection{Registered Address}
The registered address of the Association is the residential address of the Public Officer of the Association. 
\subsection{Definitions}
In these rules, the following roles are defined as:
\begin{enumerate}
    \item \textit{Director-General} means the Director-General of the Department of Fair Trading.
    \item \textit{Ordinary member} member means a member of the committee who is not an office-bearer of the Association, as referred to in rule 14 (2).
    \item \textit{Secretary} means:
    \begin{enumerate}
        \item the person holding office under these rules as secretary of the Association, or
        \item if no such person holds that office—the public officer of the Association.
    \end{enumerate}
    \item \textit{Public officer} means:
        \begin{enumerate}
            \item the person appointed by the Association under the rules of the Association.
        \end{enumerate}
\end{enumerate}
In these rules, the following terms are defined as:
\begin{enumerate}
    \item \textit{The Act} means the \underline{\href{https://legislation.nsw.gov.au/\#/view/act/1984/143/full}{Associations Incorporation Act 1984}}.
    \item \textit{The Regulation} means the \underline{\href{https://www.legislation.nsw.gov.au/\#/view/subordleg/1999/484}{Associations Incorporation Regulation 1999}}.
\end{enumerate}
In these rules:
\begin{enumerate}
    \item A reference to a function includes a reference to a power, authority and duty.
    \item A reference to the exercise of a function includes, if the function is a duty, a reference to the performance of the duty.
    \item The provisions of the Interpretation Act 1987 apply to and in respect of these rules in the same manner as those provisions would so apply if these rules were an instrument made under the Act.
\end{enumerate}

\section{Membership}
\subsection{Membership qualifications}
A person is qualified to be a member of the Association if, but only if: 
\begin{enumerate}
    %TODO: Fix references
    \item the person is a person referred to in section 15 (1) (a), (b) or (c) of the Act and has not ceased to be a member of the Association at any time after incorporation of the Association under the Act, or
    \item the person is a natural person who:
    \begin{enumerate}
       \item has completed and signed an Application for Membership form agreeing to the rules 
       and requirements of the Association, and
       \item has been approved for membership of the Association by the committee of the Association, and
       \item has paid the membership joining fee and annual membership fee within the period of two months after receipt of notification of acceptance;
    \end{enumerate}
\end{enumerate}
\subsection{Application of membership}
\begin{enumerate}
    \item \begin{enumerate}
        \item Application of a person for membership of the Association must be made in writing in the form set out in Appendix 1 to these rules, and 
        \item must be lodged with the secretary of the Association.
    \end{enumerate}
    \item As soon as practicable after receiving an application for membership, the secretary must refer the application to the committee which is to determine whether to approve or to reject the application.
    \item As soon as practicable after the committee makes that determination, the secretary must: 
    \begin{enumerate}
        \item notify the applicant that the committee approved or rejected the application (whichever is applicable), and
        \item \label{application-payment} if the committee approved the application, request the applicant to pay (within the period of two months after receipt by the applicant of the notification) the sum payable under these rules by a member as joining fee and annual subscription.
    \end{enumerate}
    \item The secretary must, on payment by the applicant of the amounts referred to in clause \ref{application-payment} within the period referred to in that provision, enter the applicant’s name in the register of members and, on the name being so entered, the applicant becomes a member of the Association.
    \item \begin{enumerate}
        \item A member may submit the name of any member who has rendered exceptional service to the Association for election at an Annual General Meeting as a Life Member of the association.
        \item A Life Member of the committee shall not have to pay any membership or annual fee, and may have any extra provision allotted to them by the committee as it sees fit.
    \end{enumerate}
\end{enumerate}
\subsection{Cessation of membership}
A person ceases to be a member of the Association if the person:
\begin{enumerate}
    \item dies, or
    \item resigns membership, or
    \item is expelled from the Association, or
    \item fails to pay the annual membership fee by the due date.
\end{enumerate}
\subsection{Membership entitlements not transferable}
A right, privilege or obligation which a person has by reason of being a member of the Association:
\begin{enumerate}
    \item is not capable of being transferred or transmitted to another person, and
    \item terminates on cessation of the person’s membership.
\end{enumerate}

\subsection{Resignation of membership}
\begin{enumerate}
    \item A member of the Association is not entitled to resign that membership except in accordance with this rule.
    \item A member of the Association who has paid all amounts payable by the member to the Association in respect of the member’s membership may resign from membership of the Association by first giving to the secretary written notice of at least one month (or such other period as the committee may determine) of the member’s intention to resign and, on the expiration of the period of notice, the member ceases to be a member.
    \item If a member of the Association ceases to be a member under clause 2, and in every other case where a member ceases to hold membership, the secretary must make an appropriate entry in the register of members recording the date on which the member ceased to be a member.
\end{enumerate}
\subsection{Register of members}
\begin{enumerate}
    \item The public officer of the Association must establish and maintain a register of members of the Association specifying the name and address of each person who is a member of the Association together with the date on which the person became a member.
    \item The register of members must be kept at the principal place of administration of the Association and must be open for inspection, free of charge, by any member of the Association at any reasonable hour, subject to the provisions of the Privacy Act.
    \item A member of the Association may obtain a copy of any part of the register on payment of a fee of \$1 for each page copied or, if some other amount is determined by the committee, that other amount, subject to the provisions of the Privacy Act.
\end{enumerate}
\subsection{Fees and subscriptions}
\begin{enumerate}
    \item A member of the Association must, on admission to membership, pay to the Association a joining fee of \$5 which will be deemed to be part of the first year’s membership fee.
   \item A member must pay to the Association an annual membership fee of \$25 or, if some other amount is determined by the committee, that other amount within two months of joining the Association: 
   \item The Association’s full financial year is 1 January to 31 December.
   \item Performing members must pay, in addition to amounts payable in clauses (1) and (2), a fee for each show in which they participate, as determined by the committee.  This must be paid no later than one month after the first rehearsal.
   \item At the discretion of the committee, the fees of an individual may be waived or reduced. A person whose fees have been waived or reduced by the committee has no voting rights.
\end{enumerate}
\subsection{Members' liabilities}
The liability of a member of the Association to contribute towards the payment of the debts and liabilities of the Association or the costs, charges and expenses of the winding up of the Association is limited to the amount, if any, unpaid by the member in respect of membership of the Association as required by rule 8.
\subsection{Resolution of internal disputes}
\begin{enumerate}
    \item Disputes between members (in their capacity as members) of the Association, and disputes between members and the Association, are to be referred to a community justice centre for mediation in accordance with the \underline{\href{https://legislation.nsw.gov.au/\#/view/act/1983/127}{Community Justice Centres Act 1983}}.
    \item At least 7 days before a mediation session is to commence, the parties are to exchange statements of the issues that are in dispute between them and supply copies to the mediator.
\end{enumerate}
\subsection{Disciplining of members}
\begin{enumerate}
    \item A complaint may be made to the committee by any person that is a member of the Association:
    \begin{enumerate}
        \item has persistently refused or neglected to comply with a provision or provisions of these rules, or
        \item has persistently and wilfully acted in a manner prejudicial to the interests of the Association.
    \end{enumerate}
    \item On receiving such a complaint, the committee: 
    \begin{enumerate}
        \item must cause notice of the complaint to be served on the member concerned, and
        \item must give the member at least 14 days from the time the notice is served within which to make submissions to the committee in connection with the complaint, and
        \item must take into consideration any submissions made by the member in connection with the complaint.
    \end{enumerate}
    \item The committee may, by resolution, expel the member from the Association or suspend the member from membership of the Association if, after considering the complaint and any submissions made in connection with the complaint, it is satisfied that the facts alleged in the complaint have been proved.
    \item The committee may, by resolution, exclude a performing member from participating in a production if that member fails to attend three consecutive rehearsals without the approval of the production team.
    \item If the committee expels or suspends a member, the secretary must, within 7 days after the action is taken, cause written notice to be given to the member of the action taken, of the reasons given by the committee for having taken that action and of the member’s right of appeal under rule 12.
    \item The expulsion or suspension does not take effect: 
    \begin{enumerate}
        \item until the expiration of the period within which the member is entitled to appeal against the resolution concerned, or
        \item if within that period the member exercises the right of appeal, unless and until the Association confirms the resolution under rule 12 (5), whichever is the later.
    \end{enumerate}
\end{enumerate}
\subsection{Right of appeal of disciplined member}
\begin{enumerate}
    \item A member may appeal to the Association in General Meeting against a resolution of the committee under rule 11, within 7 days after notice of the resolution is served on the member, by lodging with the secretary a notice to that effect.
    \item The notice may, but need not, be accompanied by a statement of the grounds on which the member intends to rely for the purposes of the appeal.
    \item On receipt of a notice from a member under clause (1), the secretary must notify the committee which is to convene a General Meeting of the Association to be held within 28 days after the date on which the secretary received the notice.
    \item At a General Meeting of the Association convened under clause (3): 
    \begin{enumerate}
        \item no business other than the question of the appeal is to be transacted, and
        \item the committee and the member must be given the opportunity to state their respective cases orally or in writing, or both, and
        \item the members present are to vote by secret ballot on the question of whether the resolution should be confirmed or revoked.
    \end{enumerate}
    \item If at the General Meeting the Association passes a special resolution in favour of the confirmation of the resolution, the resolution is confirmed.
\end{enumerate}
\section{The Committee}
\subsection{Powers of the committee}
\end{document}
